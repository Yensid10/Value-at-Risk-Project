\documentclass{article}
\usepackage{longtable}
\usepackage{xcolor}
\usepackage{enumitem}
\usepackage{geometry}
\usepackage{float}
\usepackage{url}

\newcommand\ytl[2]{
    \parbox[b]{12em}{\hfill{\color{cyan}\bfseries\sffamily #1}~$\cdots\cdots$~}\makebox[0pt][c]{$\bullet$}\vrule\quad \parbox[c]{10cm}{\vspace{6pt}\color[RGB]{20, 20, 90}\raggedright\sffamily #2\par}\\[-2pt]
}


\title{Value at Risk}
\author{Benjamin Shearlock}

\begin{document}

\begin{titlepage}
  \begin{center}
    \vspace*{1cm}
    {\LARGE \textbf{Value at Risk} \par} 
    \vspace{1.5cm}
    {\Large \textit{Project Plan} \par} 
    \vspace{0.5cm}
    {\Large BSci Final Year Project \par}
    \vspace{2cm}
    {\large \textbf{Author} \par}
    {\large Benjamin Shearlock \par}
    \vspace{2cm}
    {\large \textbf{Supervisor} \par}
    {\large Dr. Volodya Vovk \par}
    \vfill
    {\large Department of Computer Science \par}
    {\large Royal Holloway, University of London \par}
  \end{center}
\end{titlepage}


% \section{Abstract}
% This is going to be where my abstract goes. I am looking forward to writing about value at risk in this Abstract.

\section{Timeline}
Due to unfortunate circumstances, I've had rough delays to the start of this Project, so Weeks 1-3 will be noted as being a vague learning of the fundamentals behind Value at Risk. I will express this and what I hopefully plan to accomplish with the Project in the timeline below. In the first term, I want to research and create a working program to compute Value at Risk for small portfolios, that has a serviceable GUI that can be expanded on later. I will also make sure to have amply researched about back-testing and how to incorporate it into my program in some capacity. For the second term, I will research and implement applying Value at Risk for a portfolio of derivatives, as well as looking into using the Monte Carlo simulation and allowing for the computation of all this with as many stocks as necessary. I will finalise the GUI and plan to look into completing some of the extensions provided for the project, however this will depend on the overall developmental scope of the project at the time, so they will not be specified here.

\subsection{Term 1}
\vspace{-2\baselineskip}
\begin{table}[H]
  \centering
  \color{black}
  \begin{longtable}{p{1\linewidth}}
      \endfirsthead
      \endhead
      \ytl{Weeks 1-3}{
        \textbf{Project Research}
          \vspace{8pt}
          \begin{itemize}
              \item Research Value at Risk
              \item Research best coding language to use
              \item Familiarize myself with LaTeX and prepare IDE \& Git for Project specified use
          \end{itemize}
      } 
  \end{longtable}
  \end{table}

\section{Key Risks}
When developing any form of software or researching into any given field, almost every single factor, from yourself, to the environment around you, to the state of the planet to an extent, everything must come together and be balanced properly in order for you to reach your goal. This is why it is important to identify the key risks that could potentially hinder any of the many given factors for the project, and to plan for them as accordingly as possible. I will be identifying many of these risks below, ordering them with the ones I find the most important for my specific project at the top, and the ones I find will impact my project the least at the bottom.

\subsection{Personal Health and Mental Well-being}
I have this as my most likely risk as I believe it is the most applicable to what I have encountered in the past when being tasked with any assignment over an elongated period of time, such as this Project is. I have personally had various on and off health complications in the past, which can usually be manageable whilst I am able to continue working on said tasks, but it also comes in tandem with mental health challenges that makes this duality of problems a large hindrance to my productivity in general. Since this has been a norm for me over these last years, I understand its significance and worked towards setting up the best support around me to help mitigate the factors when I can, however for a Project as important as this, it will also involve keeping myself away from anything that could pose at any risk to my health, to make sure that I can continue to work on the Project to the best of my ability.

\subsection{Poor Planning and Time Management}
Other then the very personally specific problem ascertained above, the most important factor that most will find when conducting any form of large time-frame project will be the ability to manage the time they have been given. In this instance, there are specific deadline and goals throughout the project, so there is a finite time that can be used to complete the required tasks. As a requirement within this project to be able to maximise this more effectively, we have to plan our own timeline (as has already been presented in [timeline section]). Depending on feedback that will be given, it can be ensured that I am aware that I have planned properly and its up to me as an individual to stick to it, to the best of my ability, since if I cannot manage my time properly, then it is highly likely the Project cannot be completed to a satisfactory capacity.

\subsection{Emotional Health and Relationship Impact}
This risk is another one that I emphasise only due to the my own understanding of myself and how it will relate to this project. Humans are all different, they all seek different meaning to their lives, for some it is in the pursuit of knowledge, for others it is to succeed in a physical discipline, for myself it is to experience life with another person. But since mine involves that of two different people, even if I take actions to try and mitigate how it can affect me and this project, sometimes the nature of those around you can influence you in ways you cannot control. Since I am acutely aware of this danger and its effects, I will take the steps to try and separate myself from these situations as much as possible, even if it does mean to go against what I value, since that is the only solution I can see to this problem, but I am still aware that it is a risk that I cannot fully control, since this could involve familial issues as well...

\subsection{Software Risk}
I will be using various software's on my device to be able to research and create programs for this project, all of which could at some point, either due to a design flaw already found within, one that is introduced via an update or an external factor within the digital environment of the device, have issues that could potentially cause problems for the project. The main ones I am worried about are if I have problems with my IDE (VSCode) and with my Git interface (Github Desktop), but to mitigate these factors, I have already looked into and understand how I can continue this project with alternatives (Eclipse \& Git Bash for example). Even in the event of the VCS failing (Git), I am making sure that all work completed for this project is also backed up in cloud storage alongside it, as well as another alternative (SVN).

\subsection{Hardware Risk}
Since I am completing this project with the files being physically stored on my device, there is always the risk that it's hardware itself could fail, which could result in the loss of all the work completed for the project. To make sure this does not happen, I have already set up a cloud storage system that will automatically back up all the files I am working on (previously mentioned above) and I have other devices that I know are perfectly capable of being able to continue the project if needed.

\subsection{Graphical User Interface (GUI) Design Challenges}
Since this project is being presented to and assessed by others outside of just myself, it is important that the GUI is designed in a way that is easy to understand and use, as well as being aesthetically pleasing. This does involve being able to not only understand the fundamentals of GUI design, but be able to creatively display them. I am someone that has always struggled with visual creativity, I have much more confidence in the logical functioning of my endeavours, so to mitigate this lack of confidence, I will be researching into GUI design and make reference to how others perceive good GUI design to ensure that even if I cannot express much creative input, the final result is still satisfactory.

\subsection{Testing Challenges}
Testing is highly useful, as it allows for the developer to be able to ensure that the program is working as intended. This can be done in different ways, but each have their own possible drawbacks, Test Driven Development (TDD) can ensure the code will always work for the tests but can be time consuming and not always be able to test everything, testing only at the end can be quicker but can lead to difficulty in finding the source of the problem if it is not working as intended. I will be using a combination of both to ensure that I am able to write tests that are not only necessary but happen frequent enough to ensure that the program is producing intended outcomes, whilst also making sure that I am not spending too much time on testing and not enough on the actual program itself.

\subsection{Lack of Financial Knowledge}
I am aware that this is the first time I have ever looked into the financial side of mathematics/computation, so I am initially not expecting to be able to understand everything that I will be researching into, but I aim to give myself ample opportunity \& resources to understand the fundamentals of finances and the terminology used within it, so that I can comprehend what I am reading and be able to apply it to my project successfully. I will also be making sure to ask for help from my supervisor and other individuals with significant knowledge of the discipline to ensure that I am not making any mistakes in my understanding of the financial sector.

\subsection{Balancing Reporting and Coding}
When completing a project such as this, one must be able to balance the time spent on the actual coding of the program and the time spent on the reporting of the project. I am well aware that I can get lost in the coding aspect of such a task, since it is what I have always found the most fulfilling, but that will not help me in the long run if I do not have the reporting to express my work. To mitigate this, I am using LaTeX, so that formatting and presentation is mostly less stressful for me, and I can just focus on the content of the report, as well as following any previous plans I have made for the project to ensure that I am not spending too much time on one aspect of the project. I have also managed to integrate LaTeX functionality into my IDE, so that I can write the report and code in the same place, which will help me be able to balance the two together.

\subsection{Online Resource Dependence}
Since my program is aiming to take data from an online recourse (Yahoo Finance), there is always the risk that the access to this data could be impacted in some manor, such as the company/website being terminated, which from what I can access online does not appear to be likely. To combat this possible situation, I have explored for a substitute recourse that I know would still have the same utility I need for my project (Alpha Vantage).

\subsection{Efficiency Concerns}
When running any program, you will want to it run as well as possible, ensuring this through the use of efficient code. Since I have been able to do this before in the past, I believe that this is less likely to occur, especially since I will be using a language that I am not only familiar with, but is also know for its efficiency (Python). I will also be making sure to research into the best methods to ensure that my code is as efficient as possible.

\subsection{Portability Considerations}
Since my aim for this project is to create a semi-deployable product in the end (not one that is necessarily ready for commercial use, but one that can be used by others), I will need to make sure that the program is able to be used on other devices and operating systems, not just my own. I have made sure to use a language that can be run on multiple platforms (Python) and will be making sure to test the program on other devices and systems so ensure its usability.

\subsection{Geological Factors}
This is very low on this list as I will be completing this project from my home and university, both situated in a country (England) that is not known for its geological activity, so I do not believe that this will be a factor that will impact my project in any particularly significant way. I will be making sure to keep my work on the cloud just in case, as flooding, a rock slide, other forms of damage from storms or the cold could all have a certain possibility of impacting the project in some manor, but I do not believe that it is likely to occur.

\subsection{Self-Evaluation of Project Completion Ability}
To finally note, I may just not have the personal ability, whether that's a combination of any of the previous factors or just my overall intelligence, to be able to complete this project to a satisfactory level. I am aware that this is a possibility, but I am also aware that its the point of doing the project in the first place, to estimate my own individual ability. I will still make sure to utilise the recourse I have available to help where I can, and I will also express that I have confidence in my ability to complete this project in a manor that I can be proud of...

\section{Bibliography}
\begin{enumerate}
  \item Alexander, C. (2008). *Market Risk Analysis: Value-at-Risk Models.* Hoboken, NJ: Wiley. [Online] Available at: \url{https://ebookcentral-proquest-com.ezproxy01.rhul.ac.uk/lib/rhul/reader.action?docID=416450}.
  \\\textit{This is some text underneath, possibly an explanation of the reference.}
  
  \item Arbuckle, D. (2017). *Daniel Arbuckle’s Mastering Python: Build Powerful Python Applications.* Birmingham, England: Packt. [Online] Available at: \url{https://learning.oreilly.com/library/view/daniel-arbuckles-mastering/9781787283695/?ar=}.
  \\\textit{This is some text underneath, possibly an explanation of the reference.}

  \item Choudhry, M. (2006). *An Introduction to Value-at-Risk.* Chichester: John Wiley \& Sons Limited. [Online] Available at: \\ \url{https://learning.oreilly.com/library/view/an-introduction-to/9780470017579/}.
  \\\textit{This is some text underneath, possibly an explanation of the reference.}
  
  \item Duffie, D. and Pan, J. (2019). *An Overview of Value at Risk.* [Online] Available at: \url{http://web.mit.edu/people/junpan/ddjp.pdf}.
  \\\textit{This is some text underneath, possibly an explanation of the reference.}
  
  \item Föllmer, H. and Schied, A. (2016). *Stochastic Finance: An Introduction in Discrete Time.* Berlin: de Gruyter.
  \\\textit{This is some text underneath, possibly an explanation of the reference.}
  
  \item Hull, J.C. (2008). *Options, Futures, and Other Derivatives.* Upper Saddle River, NJ: Prentice Hall.
  \\\textit{This is my main reference for the project, as it is the main textbook suggested. It has a relevant chapter on Value at Risk, which explores many of the different aspects that I will need to look into for this project.}

  \item Pritsker, M. (1997). "Evaluating Value at Risk Methodologies: Accuracy versus Computational Time." *Journal of Financial Services Research* 12: 201-242. [Online] Available at: \url{https://doi.org/10.1023/A:1007978820465}.
  \\\textit{This is some text underneath, possibly an explanation of the reference.}

  \item Raman, K. (2015). *Mastering Python Data Visualization: Generate Effective Results in a Variety of Visually Appealing Charts Using the Plotting Packages in Python.* Birmingham: Packt Publishing Limited. [Online] Available at: \url{https://learning.oreilly.com/library/view/mastering-python-data/9781783988327/?ar=}.
  \\\textit{This is some text underneath, possibly an explanation of the reference.}

  \item Ulloa, R. (2015). *Kivy - Interactive Applications and Games in Python - Second Edition.* Packt Publishing. [Online] Available at: \url{https://learning.oreilly.com/library/view/kivy-interactive/9781785286926/?ar=}.
  \\\textit{This is some text underneath, possibly an explanation of the reference.}

  \item Weiming, J.M. (2015). *Mastering Python for Finance: Understand, Design, and Implement State-of-the-Art Mathematical and Statistical Applications Used in Finance with Python.* Birmingham, England: Packt Publishing. [Online] Available at: \url{https://learning.oreilly.com/library/view/mastering-python-for/9781784394516/}.
  \\\textit{This is some text underneath, possibly an explanation of the reference.}
\end{enumerate}



% \section{Acronyms/Glossary/References}
% These need to be included as well, if necessary.

\end{document}
